\documentclass[openany, 12pt]{memoir}
\usepackage[utf8]{inputenc}
\usepackage[T1]{fontenc}
\usepackage{hyphsubst}
\usepackage[english, french]{babel}

%%% Utility packages
% Links
\usepackage{hyperref}
\hypersetup{
	colorlinks,
	linkcolor = blue!30!black,
	citecolor = blue!30!black,
	urlcolor = purple!40!black
}
\usepackage{todonotes}
% Bibliography
\usepackage{cite}
\bibliographystyle{plain}
\usepackage{tocbibind}
% Glossary
\usepackage[toc,automake]{glossaries}
\newglossaryentry{test}{
	name={Test},
	description={Ceci est un test}
}
\glsaddall[types={admin,tech}]
\makeglossaries

%%% Display packages
\usepackage{graphicx}
\usepackage{fullpage}
\usepackage{xcolor}
% Headers
\setlength{\headheight}{16pt}
% Font
\usepackage{helvet}
\usepackage{inconsolata}
\renewcommand{\familydefault}{\sfdefault}

%%% Custom commands
% Unnumbered in toc
\newcommand\chapters[1]{
	\chapter*{#1}
	\addcontentsline{toc}{chapter}{#1}
}

\newcommand\anneximage[3]{
	\begin{figure}
		\centering
		\label{#1}
		\includegraphics[angle=#2,height=\textheight,width=\textwidth,keepaspectratio]{images/#1.png}
		\caption{#3}
	\end{figure}
}

\newenvironment{centerall}
    {\begin{center}\vspace*{\fill}}
    {\vspace*{\fill}\end{center}}

% DOCUMENT
\begin{document}
\frontmatter
{
	% Header
	\begin{minipage}{0.54\textwidth}
		\large
		Mémoire de PFE
		
		\vspace{10pt}
		Formation FIL
		
		\vspace{10pt}
		IMT Atlantique
	\end{minipage}
	\begin{minipage}{0.45\textwidth}
		\raggedleft
        \includegraphics[width=\textwidth]{images/header.png}
	\end{minipage}

	% Title
	\centering
    \begin{minipage}{\textwidth}
        \centering
        \vspace{20pt}
        \Large \textbf{Amélioration des performances de détection automatique d'anomalies dans des logs applicatifs}
        
        \vspace{5pt}
        \Large Oscar Gloaguen
        
        \vspace{60pt}
        Direction Générale des Finances Publiques
    \end{minipage}
    
    % Logos
    \begin{figure}[b]
        \centering
        \includegraphics[width=\textwidth]{images/logos.png}
    \end{figure}
}

\tableofcontents

\newpage
\chapters{Remerciements}
Je souhaite remercier la Direction Générale des Finances Publiques et tout particulièrement Olivier Blanc, pour m'avoir permis d'effectuer mon alternance dans cette équipe et pour m'avoir accompagné tout au long jusqu'au PFE.

\bigskip
Je tiens a remercier personnellement Robin Gries, qui a su être un vrai coéquipier tout au long de ce projet malgré sa complexité.

\bigskip
Je voudrais aussi remercier IMT Atlantique ainsi que ses professeurs, et surtout mon tuteur pédagogique Thomas Ledoux qui s'est mis a ma disposition pour le PFE.

\newpage
\begin{abstract}
\addcontentsline{toc}{chapter}{Résumé}
Les \glspl{log} sont une source importante de données détaillant le fonctionnement interne d'une application, mais ne sont pourtant que rarement utilisés a leur plein potentiel. Dans ce mémoire, je vais détailler le processus d'évolution d'un outil d'\gls{ml} qui utilise les \glspl{log} pour détecter et même tenter de prévoir des anomalies logicielles. Le projet s'apparentant plus a un projet de recherche, la démarche scientifique sera détaillée, ainsi que les caractéristiques techniques et le déroulement de l'implémentation. L'organisation du projet avec un stagiaire et moi-même sera développée, ainsi que ses impacts humains et économiques.
\end{abstract}

{\selectlanguage{english} 
\begin{abstract}
\Glspl{log} are an important source of data when it comes to the internal workings of software, but they are rarely used to their full potential. In this memoir, I will explain the evolution of a machine learning tool which uses \glspl{log} to detect and even attempt to predict software anomalies. The project being similar to a research project, the scientific protocol will be detailed, as well as the technical characteristics and the course of the implementation. Project management with an intern and myself will be developed, as well as the human and economic impacts of the project.
\end{abstract}}

\paragraph{Mots-clés}
\gls{TAL}, apprentissage automatique, apprentissage profond, analyse de logs applicatifs

% Main content
\mainmatter
\glsresetall
\chapters{Introduction}
\todo[inline]{1-2 pages}

Le projet \texttt{analyse-logs} a été développé par deux précédents apprentis de la \gls{DGFiP}, Rémi puis Léa. Cependant, les performances des algorithmes implémentés n'étant pas satisfaisantes, c'est ici que nait le sujet de ce PFE. Ce mémoire détaille le processus d'évolution de ce projet dans l'objectif d'amélioration des performances.

\bigskip
J'ai travaillé sur cette problématique en binôme avec Robin, un stagiaire en dernière année de master. Ce document touchera aussi sur l'organisation du sujet entre nous ainsi que les bénéfices et difficultés a travailler en équipe.

\bigskip
Ce sujet de PFE s'intègre dans la stratégie d'innovation du SI de la \gls{DGFiP}. L'objectif est de montrer l'efficacité de ces outils, pour mettre en valeur l'innovation et pousser leur utilisation au sein des bureaux. Dans ce cadre, une structure proche de celle d'un projet de recherche a été suivie. Pour cela, nous avons d'abord composé et étudié un état de l'art des algorithmes de détection d'anomalies existants. Ils utilisent pour la plupart des méthodes d'\gls{ml}, avec des algorithmes classiques ou de l'\gls{deep}.

\cite{deeplog}

\newpage
\chapter{Le projet \texttt{analyse-logs}}

\section{Contexte du projet}

La \gls{DGFiP} est une administration française née de la fusion de la Direction \gls{DGI} et de la \gls{DGCP} en 2008. Elle hérite alors des missions des deux entités, en faisant un service public très étendu, en charge notamment de la collecte des impôts et taxes et de la législation fiscale.

Cette administration, comme toute grande structure aujourd'hui, possède un besoin très fort en technologies de l'information, qui est rempli par le \gls{SI}

\bigskip


\section{Structure}

\subsection{DeepLog}

\subsection{Autres algorithmes}

\subsection{Blockly}

\newpage
\chapter{Recherche et analyse scientifique}

\section{État de l'art}

\section{Analyse des données}

\newpage
\chapter{Implémentation des solutions}

\section{Développement du nouveau projet}

\section{Organisation}

\newpage
\chapter{Projections de l'impact du projet}

\newpage
\chapters{Conclusion}

% BIBLIOGRAPHIE
\newpage
\backmatter
\pagenumbering{gobble}
\bibliography{biblio}

% GLOSSAIRES
\printglossary[type=admin,nonumberlist]
\newpage
\printglossary[type=tech,nonumberlist]

% ANNEXES
\newpage
\appendix
\begin{centerall}
	\textbf{\HUGE Annexes}
	\addcontentsline{toc}{chapter}{Annexes}
\end{centerall}

\anneximage{orga}{90}{Organigramme des services de la \gls{DGFiP}}

\anneximage{orga_si}{90}{Organigramme des bureaux du \gls{SI}}

\end{document}
