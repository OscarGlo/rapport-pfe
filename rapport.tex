\documentclass[openany, 12pt]{memoir}
\usepackage[utf8]{inputenc}
\usepackage[T1]{fontenc}
\usepackage{hyphsubst}
\usepackage[english, french]{babel}

%%% Utility packages
% Links
\usepackage{hyperref}
\hypersetup{
	colorlinks,
	linkcolor = blue!30!black,
	urlcolor = purple!40!black,
	citecolor = black
}
\usepackage{todonotes}
% Bibliography
\usepackage{cite}
\bibliographystyle{plain}
\usepackage{tocbibind}
% Glossary
\usepackage[toc,automake]{glossaries}
\newglossaryentry{test}{
	name={Test},
	description={Ceci est un test}
}
\makeglossaries

%%% Display packages
\usepackage{graphicx}
\usepackage{fullpage}
\usepackage{xcolor}
% Title
\usepackage{titling}
% Headers
\setlength{\headheight}{16pt}
% Font
\usepackage{helvet}
\usepackage{inconsolata}
\renewcommand{\familydefault}{\sfdefault}

%%% Custom commands
% Unnumbered in toc
\newcommand\chapters[1]{
	\chapter*{#1}
	\addcontentsline{toc}{chapter}{#1}
}

\newenvironment{centerall}
    {\begin{center}\vspace*{\fill}}
    {\vspace*{\fill}\end{center}}

% -Page de garde
% -Sommaire
% -Remerciements
% -Synthèse en anglais
% -Liste des mots-clés
% -Corps du mémoire
% -Conclusion
% -Bibliographie
% -Lexique: le lexique peut être intégré dans les annexes
% -Liste des annexes

\begin{document}
\frontmatter
\begin{titlingpage}
	% Header
	\begin{minipage}{0.55\textwidth}
		\large
		Mémoire de PFE
		
		\vspace{10pt}
		Formation FIL
		
		\vspace{10pt}
		IMT Atlantique
	\end{minipage}
	\begin{minipage}{0.45\textwidth}
		\raggedleft
        \includegraphics[width=\textwidth]{header.png}
	\end{minipage}

	% Title
	\centering
    \begin{minipage}{\textwidth}
        \centering
        \vspace{20pt}
        \Large \textbf{Amélioration des performances de détection automatique d'anomalies dans des logs applicatifs}
        
        \vspace{5pt}
        \Large Oscar Gloaguen
        
        \vspace{60pt}
        Direction Générale des Finances Publiques
    \end{minipage}
    
    % Logos
    \begin{figure}[b]
        \centering
        \includegraphics[width=\textwidth]{logos.png}
    \end{figure}
\end{titlingpage}

\tableofcontents
\newpage

\chapters{Remerciements}
\newpage

\begin{abstract}
\addcontentsline{toc}{chapter}{Résumé}
Les \glspl{log} sont une source importante de données détaillant le fonctionnement interne d'une application, mais ne sont pourtant que rarement utilisés a leur plein potentiel. Dans ce mémoire, je vais détailler le processus d'évolution d'un outil d'\gls{ml} qui utilise les \glspl{log} pour détecter et même tenter de prévoir des anomalies logicielles. Le projet s'apparentant plus a un projet de recherche, la démarche scientifique sera détaillée, ainsi que les caractéristiques techniques et le déroulement de l'implémentation. L'organisation du projet avec un stagiaire et moi-même sera développée, ainsi que ses impacts humains et économiques.
\end{abstract}

{\selectlanguage{english} 
\begin{abstract}
\Glspl{log} are an important source of data when it comes to the internal workings of software, but they are rarely used to their full potential. In this memoir, I will explain the evolution of a machine learning tool which uses \glspl{log} to detect and even attempt to predict software anomalies. The project being similar to a research project, the scientific protocol will be detailed, as well as the technical characteristics and the course of the implementation. Project management with an intern and myself will be developed, as well as the human and economic impacts of the project.
\end{abstract}}

\paragraph{Mots-clés}
\gls{TAL}, apprentissage automatique, apprentissage profond, analyse de logs applicatifs

% Main content
\mainmatter
\glsresetall
\chapters{Introduction}
\todo[inline]{1-2 pages}

Le projet \texttt{analyse-logs} a été développé par deux précédents apprentis de la \gls{DGFiP}, Rémi puis Léa. 

\medskip
Ce sujet de PFE s'intègre dans la stratégie d'innovation du SI de la \gls{DGFiP}. L'objectif est de montrer l'efficacité de ces outils, pour mettre en valeur l'innovation et pousser leur utilisation au sein des bureaux. Dans ce cadre, une structure proche de celle d'un projet de recherche a été suivie. Pour cela, nous avons d'abord composé et étudié un état de l'art des algorithmes de détection d'anomalies existants. Ils utilisent pour la plupart des méthodes d'\gls{ml}, avec des algorithmes classiques ou de l'\gls{deep}.

\cite{deeplog}

\newpage
\chapter{Le projet \texttt{analyse-logs}}

\section{Historique}

\subsection{Travail de Rémi}

\subsection{Travail de Léa}

\section{Structure}

\subsection{DeepLog}

\subsection{Autres algorithmes}

\subsection{Blockly}

\newpage
\chapter{Recherche et analyse scientifique}

\section{État de l'art}

\section{Analyse des données}

\newpage
\chapter{Implémentation des solutions}

\section{Développement du nouveau projet}

\section{Organisation}

\newpage
\chapter{Projections de l'impact du projet}

\newpage
\chapters{Conclusion}

\newpage
\backmatter
\pagenumbering{gobble}
\bibliography{biblio}
\newpage

\printglossary[nonumberlist]
\end{document}