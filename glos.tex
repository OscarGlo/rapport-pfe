% Acronym with description
\newcommand{\acronymdesc}[4]{
	\newacronym[type=#1,description={#3: #4}]{#2}{#2}{#3}
}

\newglossary[ad]{admin}{adm}{ado}{Glossaire administratif}
\newglossary[tc]{tech}{tch}{tco}{Glossaire technique}

% ADMINISTRATIF
\acronymdesc{admin}{DGFiP}
	{Di\-rec\-tion Gé\-né\-rale des Fi\-nan\-ces Pu\-bli\-ques}
	{Service public de l'État rattaché au ministère des finances, chargé des missions de gestion publique et de fiscalité}

\acronymdesc{admin}{DGCP}
	{Di\-rec\-tion Gé\-né\-rale des Comptes Pu\-blics}
	{Aussi appelée <<trésor public>>, ancienne administration chargée de la gestion des comptes de l'État et du recouvrement des impôts}
	
\acronymdesc{admin}{DGI}
	{Di\-rec\-tion Gé\-né\-rale des Impôts}
	{Ancienne administration chargée de la liquidation des impôts}
	
\acronymdesc{admin}{SI}
	{Ser\-vice des sys\-tèmes d'In\-for\-ma\-tion}
	{Service support de la \gls{DGFiP} chargé de la gestion informatique et du développement d'applications}
	
\acronymdesc{admin}{DPN}
	{Dir\-ec\-tion des pro\-jets nu\-mé\-ri\-ques}
	{Direction au sein du \gls{SI} regroupant les différents bureaux en charge de la direction et de la réalisation de projets dans le numérique}

\acronymdesc{admin}{BSI-3}
	{Bu\-reau du SI des pro\-fes\-sion\-nels}
	{Auparavant nommé SI-1C, bureau dépendant de la \gls{DPN}, chargé du développement et de la maintenance des applications du domaine de la fiscalité des professionnels}

\acronymdesc{admin}{MEDOC}
	{ME\-ca\-ni\-sa\-tion Des Opé\-ra\-tions Comp\-tables}
	{Application traitant l'encaissement des taxes (telles que la TVA) ainsi que la génération d'écritures comptables pour l'État}

\acronymdesc{admin}{DTT}
	{Di\-vi\-sion tech\-nique trans\-verse}
	{Ancienne division du SI-1C (\gls{BSI-3} aujourd'hui) apportant un soutien technique aux autres équipes et proposant des projets d'innovation}

\newglossaryentry{transverse}{
    name={trans\-ver\-se},
	type=admin,
    description={Utilisé fréquemment a la \gls{DGFiP} comme synonyme de transversal, dans le sens ``recoupant plusieurs disciplines ou secteurs''}
}

% TECHNIQUE
\acronymdesc{tech}{TAL}
	{trai\-te\-ment au\-to\-ma\-tique du lan\-gage}
	{Ensemble de méthodes visant à analyser et traiter le langage naturel}

\newglossaryentry{log}{
    name={log},
	type=tech,
    description={De l'anglais log (journal), sortie d'une application (souvent un fichier) représentant le chemin d'exécution d'une application (voir section \ref{logs})}
}

\newglossaryentry{ml}{
    name={ap\-pren\-tis\-sage au\-to\-ma\-tique},
	type=tech,
    description={Ou \textit{machine learning} en anglais, ensemble de méthodes visant a développer des algorithmes généraux basés sur l'apprentissage de données, s'opposant a un algorithme explicite classique}
}

\newglossaryentry{deep}{
    name={ap\-pren\-tis\-sage pro\-fond},
	type=tech,
    description={Branche de l'\gls{ml} se basant sur des réseaux de neurones artificiels, comportant plusieurs couches de traitement de données permettant d'extraire des caractéristiques complexes}
}

\newglossaryentry{LSTM}{
    name={LSTM},
	type=tech,
    description={Long Short Term Memory (Longue mémoire à court terme): Type de réseau de neurones en \gls{deep} capable de retenir de l'information sur une longue durée}
}